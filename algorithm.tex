\section{Algorithm}
\label{section:algorithm}

\mbsays{Please have a go at the text and finish where unfinished}

This section gives a semi-decision procedure for polynomial integer
equations.
%
It is based on the following two elementary observations:
%
\begin{enumerate}
\item{If a set of equations is satisfiable over the integers then it
  will be satisfiable modulo any number.  We use the converse -- if a
  set of equations is unsatisfiable over a specific modulus then it
  must be unsatifiable over the integers.}
\item{If a set of equations is solvable modulo $p$ and $q$ with $p$ and
  $q$ are coprime then they will be solvable modulo $p*q$.  This is a
  consequence of the Chinese Remainder Theorem.}
\end{enumerate}
%
These are used to create an abstraction-refinement loop using a pair
of solvers.  One computes the solutions modulo a product of primes
(giving an over-approximation of the true satisfying assignments) and
the other checks these candidates.
%
The algorithm consists of a loop of four phases, illustrated in Figure
\ref{figure:algorithm}

\begin{description}
  \item[Pick Prime]{}
  \item[Solve Modulo Subproblems]{}
  \item[Extract Candidates]{}
  \item[Check Candidates]{}
\end{description}
\mbsays{Fill in the description}

\begin{figure*}
  \caption{A semi-decision procedure for polynomial integer equations.}
  \label{figure:algorithm}.
\end{figure*}
\mbsays{Create a flow-chart of the phases.  I can do this if you don't
feel like doing it}



\mbsays{If have a worked example and want to include it, here would be
  good.}


We should probably say something about correctness, complexity or completeness.
That might be ``We don't know''. \mbsays{Think about this.}

I think there are theorems which says something like:
\begin{itemize}
\item{This is a semi-decision procedure (if it terminates then it is correct)}
\item{If all of the variables are bounded so they have a finite range
  then the algorithm will terminate}
\end{itemize}
but I am unsure if we will have time to prove these.

