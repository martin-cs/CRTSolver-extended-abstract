\section{Algorithm}
\label{section:algorithm}

\begin{figure}[th]
  \begin{center}
  \begin{tikzpicture}
  
  \path
    let
     \n{vsep} = {2cm},
     \n{hsep} = {2cm}
    in
     coordinate (selectmod)  at (-1*\n{hsep},  1*\n{vsep})
     coordinate (solvemod)   at (-1*\n{hsep}, -1*\n{vsep})
     coordinate (candidates) at ( 1*\n{hsep}, -1*\n{vsep})
     coordinate (check)      at ( 1*\n{hsep},  1*\n{vsep})
     coordinate (sat)        at ( 3*\n{hsep},  1*\n{vsep})
     coordinate (unsat)      at (-3*\n{hsep}, -1*\n{vsep})
     coordinate (tr)         at ( 2*\n{hsep},  2*\n{vsep})
     coordinate (tl)         at (-2*\n{hsep},  2*\n{vsep})
     coordinate (br)         at (-2*\n{hsep}, -2*\n{vsep})
     coordinate (tm)         at ( 0*\n{hsep},  2*\n{vsep})
     coordinate (bm)         at ( 0*\n{hsep}, -2*\n{vsep})
     coordinate (form)       at ( 0*\n{hsep}, 2.5*\n{vsep})
     coordinate (fleft)      at (-1.75*\n{hsep},1.65*\n{vsep})
     coordinate (fright)     at ( 1.75*\n{hsep},1.65*\n{vsep})
    ;


  % Fill inside
  \path [pattern color=red, pattern=dots] (br) -| (tm) -| (br);
  \path [pattern color=blue!40, pattern=crosshatch] (tr) -| (bm) -| (tr);

  \tikzset{labelstyle/.style = {shape=rectangle, fill=white, minimum width=2.2cm, minimum height=0.55cm}}
  \node [anchor=north west, labelstyle] at (tl) {Modulo Solver};
  \node [anchor=north east, labelstyle] at (tr) {Integer Solver};

  % Outside box
  \draw (tr) -| (br) -| (tr);
  
  \tikzset{phasestyle/.style = {draw, shape=rectangle, minimum width=2cm, minimum height=2cm, fill=white}}
     
  \node [phasestyle] (Selectmod) at (selectmod) {Select};
  \node [phasestyle] (Solvemod) at (solvemod) {Solve};
  \node [phasestyle] (Candidates) at (candidates) {Candidates};
  \node [phasestyle] (Check) at (check) {Check};

  % Results
  \tikzset{resultstyle/.style = {shape=ellipse}}%draw, minimum width=1.5cm}}

  \node [resultstyle] (Sat) at (sat) {SAT};
  \node [resultstyle] (Unsat) at (unsat) {UNSAT};

  \tikzset{resultstyle/.style = {draw, -triangle 60}}
  
  \path [resultstyle] (Solvemod.west) -> (Unsat.east);
  \path [resultstyle] (Check.east) -> (Sat.west);
  
  % Connecting arrows
  \tikzset{arrowstyle/.style = {-triangle 60, draw}}

  \path [arrowstyle] (Selectmod.south) -> (Solvemod.north);
  \path [arrowstyle] (Solvemod.east) -> (Candidates.west);
  \path [arrowstyle] (Candidates.north) -> (Check.south);
  \path [arrowstyle] (Check.west) -> (Selectmod.east);

  \path (Selectmod.north east) ++(0,1.5) coordinate (above);
  \path [arrowstyle] (above) -> (Selectmod.north east);
  
  % Formula flow
  \tikzset{formpath/.style = {draw, dashed, -triangle 60}}
  
  \node (Form) at (form) {Formula};
  \path [formpath] (Form.south) |- (fleft) |- (Solvemod.north west);
  \path [formpath] (Form.south) |- (fright) |- (Check.north east);
  
\end{tikzpicture}
    
  \end{center}
  \caption{A flow-chart description of a Abstraction/Refinement style semi-decision procedure for Diophantine equations}
  \label{figure:algorithm}
\end{figure}


This section gives a semi-decision procedure for polynomial integer
equations.
%
It is based on the following two elementary observations:
%
\begin{enumerate}
\item{If a set of equations is satisfiable over the integers then it
  will be satisfiable modulo any number.  We use the contraposition -- if a
  set of equations is unsatisfiable over a specific modulus then it
  must be unsatifiable over the integers.}
\item{If a set of equations is solvable modulo $p$ and $q$ with $p$ and
  $q$ being coprime then they will be solvable modulo $p*q$.  This is a
  consequence of the Chinese Remainder Theorem.}
\end{enumerate}
%
These are used to create an abstraction-refinement style loop using a pair
of solvers.  One computes the solutions modulo a product of primes
(giving an over-approximation of the true satisfying assignments) and
the other checks these candidates.
%
The algorithm consists of a loop of four phases, illustrated in Figure
\ref{figure:algorithm}:

\begin{description}
\item[Select]{A new modulo $m_i$, co-prime with all previous modulos is chosen.
We currently use an ascending list of primes, starting from $m_0 = 2$.}

\item[Solve]{A copy of the set of equations computed $mod m_i$ is added
  to the modulo solver.  Then the modulo solver is checked for
  satisfiability.  If it is unsatisfiable then the equations are unsolvable
  modulo in at least one of $m_i$'s and thus the original problem is
  unsatisfiable.  If it is satisfiable then the model will give
  solutions for the equations modulo every $m_i$ used so far.}

\item[Candidates]{The Chinese Remainder Theorem is used to compute
  a solution modulo $\Pi m_i$.  These are used to build a number of
  candidates over $\mathbb{Z}$, each of which is the solution
  modulo $\Pi m_i$ plus $k \Pi m_i$.  Currently we produce candidates
  for  $k \in [-2,2]$.}

\item[Check]{The original equations are evaluated using the candidate
  solutions.  If this gives a satisfying assignment then the original
  system is satisfiable.  If all of the candidates fail to satisfy the
  original equations then a new modulus must be used.}
\end{description}


\mbsays{If have a worked example and want to include it, here would be
  good.}

%We should probably say something about correctness, complexity or completeness.
%That might be ``We don't know''. \mbsays{Think about this.}
%
%I think there are theorems which says something like:
%\begin{itemize}
%\item{This is a semi-decision procedure (if it terminates then it is correct)}
%\item{If all of the variables are bounded so they have a finite range
%  then the algorithm will terminate}
%\end{itemize}
%but I am unsure if we will have time to prove these.

The algorithm above is a semi-decision procedure -- if it terminates
then it will give the correct result.
%
However the termination of the algorithm is less obvious than it might
appear.
%
There are equations which are satisfiable modulo any $m$ but are not
satisfiable over $\Z$.
For example:
%
\begin{equation*}
  w^2 + x^2 + y^2 + z^2 + 1 = 0
\end{equation*}
%
\noindent clearly as no solutions over $\Z$ but by Lagrange's Four
Square Theorem there are solutions for any modulus $m$.
