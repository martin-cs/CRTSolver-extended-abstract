\section{Introduction}

A reasonable performance floor for SMT solvers is that they should be
faster than a human solving the same problem with a pen and paper.
%
In the case of non-linear expressions over theory of integers this
floor is not always met.
%
For example consider the following problem:

\mbsays{Give, in mathematical notation, not SMT-LIB, the simplest
  problem which times out}

In our experiments we found that \mbsays{which solvers and how long to
time out} but observing \mbsays{Give some kind of hint} will allow
many humans to solve it faster.

\mbsays{Say why it is undecideable}
This has an immediate practical consequence of their being no
algorithms for the general case but we also believe that it has had
chilling effect on the theory in general.
%
We believe there are many interesting and useful fragments of the
theory of integers for which there are practical decision procedures,
semi-decision procedures or good heuristics.
\mbsays{For a full workshop paper, this would be a good place to
  discuss the existing literature and techniques.}
%
In this paper we will consider one such fragment, existential
polynomial equations.
%
In Section \ref{section:algorithm} we propose an
abstraction-refinement style algorithm that uses the Chinese Remainder
Theorem.
Section \ref{section:results} we describe some preliminary experiments
using our implementation of the algorithm.
Finally Section \ref{section:conclusion} discusses the future
potential of this approach.





