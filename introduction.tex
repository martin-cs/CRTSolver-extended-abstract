\section{Introduction}

A reasonable performance floor for SMT solvers is that they should be
faster than a human solving the same problem with a pen and paper.
%
In the case of non-linear expressions over theory of integers this
floor is not always met.
%
For example consider the following problem: 
% inline: $2x^3 + 3x^2 + 6x + 8 = 0$
\[2x^3 + 3x^2 + 6x + 8 = 0\]

In our experiments we found that cvc5 timed out after 30 
seconds.
Observing that all coefficients are positive including the
constant so $x$ must be strictly negative and then trying decreasing
values ($x = -1, x = -2, \dots$) gives an answer very quickly.

Hilbert's tenth problem asked (in modern terminology) whether 
there is an decision procedure for Diophantine equations 
(a polynomial equation possessing only integer coefficients).
%
The MRDP theorem gives a construction from recursively enumerable sets
to Diophantine equations, thus showing the undecideability of satisfiability.

This has an immediate practical consequence of their being no
algorithms for the general case but we also believe that it has had
chilling effect on the theory in general.
%
We believe there are many interesting and useful fragments of the
theory of integers for which there are practical decision procedures,
semi-decision procedures or good heuristics.
%----For a full workshop paper, this would be a good place to----%
%----discuss the existing literature and techniques----%
In this paper we will consider one such fragment, existential
polynomial equations.
%
In Section \ref{section:algorithm} we propose an
abstraction-refinement style algorithm that uses the Chinese Remainder
Theorem.
Section \ref{section:results} describes some preliminary experiments
using our implementation of the algorithm.
Finally Section \ref{section:conclusion} discusses the future
potential of this approach.
