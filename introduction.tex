\section{Introduction}

\mmsays{Please do check the intro and make/propose changes as necessary}

A reasonable performance floor for SMT solvers is that they should be
faster than a human solving the same problem with a pen and paper.
%
In the case of non-linear expressions over theory of integers this
floor is not always met.
%
For example consider the following problem: 
% inline: $2x^3 + 3x^2 + 6x + 8 = 0$
\[2x^3 + 3x^2 + 6x + 8 = 0\]

\mmsays{only cvc5 times out on this equation, not Z3. There are no equations
for which Z3 times out and CRTSolver doesn't}
In our experiments we found that cvc5 timed out after 30 
seconds but observing that small integers (i.e. $0$, $1$, $-1$)
result in outputs greater than or less than zero - but never exactly
zero, will allow many humans to solve it faster.

The undecidability of non-linear integer equations is a characteristic
determined by Hilbert's tenth problem and the subsequent MRDP theorem 
(also known as Matiyasevich's theorem). Hilbert's tenth problem asks 
if there is a general formula that, for a given Diophantine equation 
(a polynomial equation possessing only integer coefficients), can 
determine the existence of an integer solution.

This has an immediate practical consequence of their being no
algorithms for the general case but we also believe that it has had
chilling effect on the theory in general.
%
We believe there are many interesting and useful fragments of the
theory of integers for which there are practical decision procedures,
semi-decision procedures or good heuristics.
%----For a full workshop paper, this would be a good place to----%
%----discuss the existing literature and techniques----%
In this paper we will consider one such fragment, existential
polynomial equations.
%
In Section \ref{section:algorithm} we propose an
abstraction-refinement style algorithm that uses the Chinese Remainder
Theorem.
Section \ref{section:results} describes some preliminary experiments
using our implementation of the algorithm.
Finally Section \ref{section:conclusion} discusses the future
potential of this approach.
